\documentclass[journal]{IEEEtran}

\usepackage{amsmath}
\usepackage{amssymb}
\interdisplaylinepenalty=2500

\hyphenation{op-tical net-works semi-conduc-tor}

\begin{document}

\title{Notification-Gated Reliability Unit Commitment:\\ Least-Distortionary Out-of-Market Interventions}

\author{Author~One,~\IEEEmembership{Member,~IEEE,}
        Author~Two,
        and~Author~Three,~\IEEEmembership{Senior~Member,~IEEE}%
\thanks{Affiliations and funding acknowledgments.}%
\thanks{Manuscript received XXXX; revised XXXX.}}

\markboth{IEEE Transactions on Power Systems}%
{Authors: Notification-Gated Reliability Unit Commitment}

\maketitle

\begin{abstract}
The abstract goes here.
\end{abstract}

\begin{IEEEkeywords}
Reliability unit commitment, robust optimization, market design,
notification time, linear decision rules.
\end{IEEEkeywords}

\IEEEpeerreviewmaketitle


%% ===================================================================
%%  SECTION 1 — INTRODUCTION
%% ===================================================================
\section{Introduction}

\IEEEPARstart{R}{eliability} processes are required in power system
operations due to a
fundamental market failure in the provision of reliability. To address
this, system operators routinely commit generation resources outside of
the energy market through reliability-driven commitment actions.
While effective in preserving system feasibility, empirical evidence
and operational experience indicate that the heuristics used to trigger
these commitments are often overly conservative, leading to systematic
over-commitment and distortion of price signals in both the day-ahead
and real-time markets.

Motivated by this gap, this work proposes a robust optimization
framework to identify the minimal set of generation commitment decisions
that are strictly necessary prior to the next day-ahead market clearing
to ensure system reliability. Rather than optimizing all commitment and
dispatch decisions under uncertainty, the proposed methodology focuses
exclusively on commitment decisions that cannot be deferred without
risking infeasibility under adverse realizations of load and renewable
generation.

System reliability is enforced with respect to a worst-case forecast
model of net demand, capturing uncertainty in both load and renewable
production. Commitment decisions are selected so as to guarantee
feasibility for all realizations within this uncertainty set, while
minimizing the extent of out-of-market intervention. In this sense, the
resulting commitments may be interpreted as the least-distortionary
reliability actions, providing a principled defense of operator
interventions while preserving market signals to the greatest extent
possible.

\subsection{Contributions}

This paper makes the following contributions.

First, we develop a notification-gated formulation for reliability unit
commitment that explicitly models the interaction between physical
generator lead times and the sequencing of market and operational
decision processes. The proposed framework identifies the subset of
commitment decisions that cannot be deferred to subsequent market stages
and restricts out-of-market interventions to this non-deferrable
decision set.

Second, we introduce a minimal-intervention reliability objective that
minimizes only startup actions and associated no-load costs induced by
non-deferrable commitments. This objective decouples reliability
enforcement from full economic re-optimization, thereby preserving
market-based commitment and dispatch outcomes while ensuring worst-case
feasibility under uncertainty.

Third, we demonstrate that the proposed formulation generalizes
naturally across operational horizons, including multi-day reliability
assessment and day-ahead reliability unit commitment, enabling
coordinated treatment of long-lead thermal resources and near-term
reliability adjustments within a unified optimization framework.

Fourth, we conduct a comprehensive numerical evaluation comparing the
proposed approach against standard reliability unit commitment and
integrated market-reliability formulations. Results show significant
reductions in early commitment interventions and redispatch magnitude
without compromising system reliability.


%% ===================================================================
%%  SECTION 2 — LITERATURE REVIEW
%% ===================================================================
\section{Literature Review}

% TODO: Three threads to cover:
%
% (a) RUC in practice — ISO/RTO operational descriptions, FERC reports,
%     SPP/MISO/CAISO market manuals documenting DARUC processes.
%     Emphasize that RUC is widespread but formulations are proprietary
%     and the academic literature rarely models the sequential
%     decision structure faithfully.
%
% (b) Robust and stochastic UC — Bertsimas, Litvinov, Jiang et al.
%     Position this paper relative to the robust UC literature:
%     we are not proposing a new uncertainty model (that is paper 2),
%     but a new *decision scope* within which robustness is enforced.
%
% (c) Market design and out-of-market actions — uplift, price formation,
%     convex hull pricing literature. Brief treatment: motivation for
%     why minimizing interventions matters, even if we don't compute
%     LMPs in the case study.
%
% The gap: no existing formulation explicitly models notification
% times to restrict the reliability optimization to non-deferrable
% decisions.


%% ===================================================================
%%  SECTION 3 — THE RELIABILITY COMMITMENT PROBLEM
%% ===================================================================
\section{The Reliability Commitment Problem}

\subsection{Operational Timeline}

While all system operators have their own idiosyncrasies,
the North American RTOs share a common schematic structure in their
sequence of decision processes. For this work, we focus on the
pattern in use by Southwest Power Pool (SPP), but the outline applies
to most other system operators with the notable exception of
NYISO, which integrates the DARUC and DAMKT clearing.

For notational clarity, we refer to the Operating Day (OD) as
the day for which the operation plan is being generated.
The principal scheduling process, the Day-Ahead Market (DAMKT),
finishes solving around noon of OD$-$1. Decisions of that process
are communicated at 2 pm OD$-$1. The DA-RUC process follows this
at 5 pm. The DARUC considers a decision horizon covering
OD and OD$+$1.

% TODO: Complete the multi-day reliability assessment (MDRA) description.
% MDRA is advisory, 4-7 day horizon, commits long-lead resources.
% Then: DAMKT → DARUC → ID-RUC intraday.
% A timeline figure would be valuable here.

\subsection{Unit Commitment Formulation}

% State the canonical UC formulation ONCE. Both DAMKT and standard
% DARUC share these constraints; they differ only in objective.

The following formulation captures the core unit commitment
constraints common to both the day-ahead market and reliability
processes. Let $\mathcal{I}$ index generating resources,
$\mathcal{T} = \{1,\dots,T\}$ index operating periods,
$\mathcal{N}$ index buses, and $\mathcal{L}$ index transmission
lines.

\begin{subequations}\label{eq:uc_base}
\begin{align}
%--- power balance ---
&\sum_{i} p_{i,t} + s_t^{p} = \sum_n d_{n,t},
  \quad \forall t, \label{eq:uc_bal} \\
%--- transmission ---
&-\bar{F}^l \leq \sum_n S_n^l
  \Big(\sum_{i\in\mathcal{I}_n} p_{i,t} - d_{n,t}\Big)
  \leq \bar{F}^l,
  \quad \forall l,t, \label{eq:uc_flow} \\
%--- block dispatch ---
&p_{i,t,b} \leq \bar{P}_{i,b}\, u_{i,t},
  \;\; \forall i,t,b, \quad
  p_{i,t} = \textstyle\sum_b p_{i,t,b},
  \;\; \forall i,t, \label{eq:uc_block} \\
%--- renewable limit ---
&p_{j,t} \leq \bar{p}^{\text{fcst}}_{j,t},
  \quad \forall j \in \mathcal{I}^{\text{ren}},\; t, \label{eq:uc_ren} \\
%--- capacity ---
&\underline{P}_i\, u_{i,t} \leq p_{i,t}
  \leq \bar{P}_i\, u_{i,t},
  \quad \forall i,t, \label{eq:uc_cap} \\
%--- ramp up ---
&p_{i,t} - p_{i,t-1} \leq R_i^U (u_{i,t-1} + v_{i,t}),
  \quad \forall i,t, \label{eq:uc_ru} \\
%--- ramp down ---
&p_{i,t-1} - p_{i,t} \leq R_i^D (u_{i,t} + w_{i,t}),
  \quad \forall i,t, \label{eq:uc_rd} \\
%--- logic ---
&u_{i,t} - u_{i,t-1} = v_{i,t} - w_{i,t},
  \quad \forall i,t, \label{eq:uc_logic} \\
%--- MUT ---
&\sum_{t'=t-\text{MUT}_i+1}^{t} v_{i,t'} \leq u_{i,t},
  \quad \forall i,\; t \geq \text{MUT}_i, \label{eq:uc_mut} \\
%--- MDT ---
&\sum_{t'=t-\text{MDT}_i+1}^{t} w_{i,t'} \leq 1 - u_{i,t},
  \quad \forall i,\; t \geq \text{MDT}_i, \label{eq:uc_mdt} \\
%--- initial conditions ---
&\sum_{t'=1}^{UT_i} u_{i,t'} = UT_i, \quad
  \sum_{t'=1}^{DT_i} u_{i,t'} = 0,
  \quad \forall i, \label{eq:uc_init} \\
%--- domains ---
&\mathbf{p},\, \mathbf{s} \geq 0, \quad
  \mathbf{u},\, \mathbf{v},\, \mathbf{w} \in \{0,1\}.
  \label{eq:uc_domain}
\end{align}
\end{subequations}

We denote this shared constraint set as $\mathcal{X}^{\text{UC}}$.

\subsection{Day-Ahead Market (DAMKT)}

The DAMKT minimizes total production cost including energy,
no-load, startup, and shutdown costs over all generators and periods:
\begin{align}\label{eq:damkt_obj}
\min_{\mathbf{p},\mathbf{u},\mathbf{v},\mathbf{w},\mathbf{s}}
\quad &\sum_{i,t} \Big[
  C_i^{NL} u_{i,t} + C_i^{SU} v_{i,t} + C_i^{SD} w_{i,t}
  + \sum_b C_{i,b}^p\, p_{i,t,b}
\Big] \nonumber\\
&+ \sum_t M^p s_t^p \\
\text{s.t.} \quad
& \eqref{eq:uc_base}. \nonumber
\end{align}

\subsection{Standard Day-Ahead RUC (DARUC)}

The standard DARUC uses the same constraint set
$\mathcal{X}^{\text{UC}}$ but removes energy costs from the
objective, retaining only commitment-related costs:
\begin{align}\label{eq:daruc_obj}
\min_{\mathbf{p},\mathbf{u},\mathbf{v},\mathbf{w},\mathbf{s}}
\quad &\sum_{i,t} \Big[
  C_i^{NL} u_{i,t} + C_i^{SU} v_{i,t} + C_i^{SD} w_{i,t}
\Big]
+ \sum_t M^p s_t^p \\
\text{s.t.} \quad
& \eqref{eq:uc_base}. \nonumber
\end{align}

% TODO: Note that in practice, the DARUC may also fix u^m from the
% DAMKT solution and only allow additional commitments. This varies
% by ISO. SPP's practice is [describe].

\subsection{Why Standard DARUC Over-Commits}\label{sec:overcommit}

% TODO: This is the diagnostic argument that sets up Section 4.
% Key points:
%
% (1) Standard DARUC optimizes u_{i,t} for ALL i and ALL t in the
%     horizon. But most of these decisions will be revisited by the
%     next market clearing (DAMKT for OD+1, or ID-RUC intraday).
%     The DARUC is solving a problem that is mostly not its to solve.
%
% (2) Because the DARUC sees the full horizon and minimizes total
%     commitment cost, it may commit units early to avoid more
%     expensive startups later — even when those later decisions
%     can be deferred to a subsequent process that will have better
%     information.
%
% (3) The result: commitment decisions that are technically feasible
%     but premature. Units are started hours or days before they are
%     needed, displacing market-cleared resources and distorting
%     price signals.
%
% (4) The core issue is not that the DARUC commits the wrong units,
%     but that it commits units whose startup decisions could have
%     waited. The formulation lacks a notion of *deferrability*.


%% ===================================================================
%%  SECTION 4 — NOTIFICATION-GATED RUC (THE CONTRIBUTION)
%% ===================================================================
\section{Notification-Gated Reliability Commitment}

The proposed formulation is solved in two phases.
Phase~1 determines the minimum-cost set of non-deferrable reliability
commitments against the mean forecast.
Phase~2 verifies that these commitments are sufficient to guarantee
dispatch feasibility under worst-case uncertainty.
When Phase~2 identifies a robustness shortfall, the two phases are
iterated via column-and-constraint generation until robust feasibility
is attained (Section~\ref{sec:augment}).

\subsection{Gating Sets}\label{sec:gating}

The notification (lead) time of a generating unit determines which
commitment decisions must be finalized at the current decision stage
and which can be deferred to subsequent market or operational
processes.
Commitment decisions should be made binding only when deferral would
violate physical feasibility.

Let decision process $k$ represent the current reliability or market
commitment stage (e.g., MDRA, DA-RUC, ID-RUC), and let
$t^{\text{next}}$ denote the first operating period whose commitment
decisions can be modified by the next available process.
Let $t \in \mathcal{T}^k = \{1,\dots,T\}$ index the physical
operating periods in the look-ahead horizon considered at stage~$k$.

For generating resource~$i$ with startup notification time
$L_i^{SU}$, a startup decision affecting operating period~$t$ must
be issued at the current stage if it cannot be deferred to the next
process.
This condition defines the \emph{gating set}:
\begin{equation}\label{eq:gating_set}
    \mathcal{T}_{i}^{\text{gate}} \;:=\;
    \bigl\{\, t \in \mathcal{T}^k \;\big|\;
            t - L_i^{SU} < t^{\text{next}} \bigr\}.
\end{equation}

Periods in $\mathcal{T}_{i}^{\text{gate}}$ correspond to startup
decisions that must be initiated before the next commitment process
in order to preserve feasibility.
Startup decisions outside this set may be deferred and are excluded
from the current reliability optimization.

Because $t - L_i^{SU}$ is increasing in~$t$, the gating set is a
contiguous block of early periods: if period~$t$ is gated, then every
earlier period $t' < t$ is also gated.
Generators with longer notification times have larger gating sets,
reflecting the greater advance notice required to start them.

The gating set restricts \emph{startup} decisions ($v_{i,t}$), not
commitment status ($u_{i,t}$).
A unit already committed by the market does not require a new startup
decision and is therefore not subject to gating.

\subsection{Market-Reliability Decomposition}\label{sec:decomp}

Let $(\mathbf{u}^m, \mathbf{v}^m, \mathbf{w}^m)$ denote the
commitment schedule from the preceding DAMKT solution.
The reliability process operates on the \emph{total} commitment
variables $(u_{i,t}, v_{i,t}, w_{i,t})$, which are constrained to
preserve all market-decided commitments:
\begin{equation}\label{eq:nodecommit}
    u_{i,t} \;\geq\; u_{i,t}^m, \qquad \forall\, i,t.
\end{equation}
The startup and shutdown variables $v_{i,t}$ and $w_{i,t}$ are
determined by the logic constraint~\eqref{eq:uc_logic} applied to the
total commitment~$u_{i,t}$ and are not independently decomposed.

The \emph{incremental commitment} is defined as
\begin{equation}\label{eq:uprime}
    u'_{i,t} \;:=\; u_{i,t} - u_{i,t}^m \;\in\; \{0,1\},
    \qquad \forall\, i,t,
\end{equation}
where $u'_{i,t} = 1$ indicates a reliability intervention: a unit
brought online beyond the market solution.

A reliability startup at period~$t$ may \emph{absorb} a market
startup at a later period: if the market planned to start unit~$i$ at
period~$t{+}2$, but the reliability process starts it at~$t$, the
unit is already online at~$t{+}2$ and the market startup is
superseded.
This interaction is handled automatically by the logic
constraint~\eqref{eq:uc_logic} operating on total~$u$.

\subsection{Phase~1: Deterministic Gated LD-RUC}\label{sec:phase1}

Phase~1 identifies the minimum-cost reliability commitments that
ensure feasibility under the mean renewable forecast~$\boldsymbol{\mu}$.
The formulation uses the full constraint set
$\mathcal{X}^{\text{UC}}$ over the entire planning horizon; only the
\emph{objective} is restricted to the gating window.

\begin{subequations}\label{eq:phase1}
\begin{align}
\min_{\substack{
  \mathbf{u},\mathbf{v},\mathbf{w},\\
  \mathbf{p},\mathbf{s}}} \quad
& \sum_{i \in \mathcal{I}}
  \sum_{t \in \mathcal{T}_i^{\text{gate}}}
  \!\Big[\,
    C_i^{SU}\, v_{i,t}
    \;+\; C_i^{NL}\, u'_{i,t}
  \Big]
  \;+\; M \!\sum_{t} s_t^p
  \label{eq:phase1_obj} \\[4pt]
\text{s.t.}\quad
& \eqref{eq:uc_base}
  \quad\text{with } \bar{p}_{j,t}^{\text{fcst}} = \mu_{j,t},
  \label{eq:phase1_uc} \\
& u_{i,t} \geq u_{i,t}^m,
  \quad \forall\, i,t,
  \label{eq:phase1_lock} \\
& u'_{i,t} = u_{i,t} - u_{i,t}^m,
  \quad \forall\, i,t.
  \label{eq:phase1_uprime}
\end{align}
\end{subequations}

The objective~\eqref{eq:phase1_obj} penalizes startup costs
$C_i^{SU} v_{i,t}$ and incremental no-load costs
$C_i^{NL} u'_{i,t}$ only within the gating window
$\mathcal{T}_i^{\text{gate}}$.
Startup terms corresponding to the DAMKT solution
($v_{i,t} = v_{i,t}^m$) contribute a constant under the
no-decommitment constraint~\eqref{eq:phase1_lock} and do not
influence the optimization.
Commitment variables outside the gating window are free but carry no
objective cost; the optimizer adds non-gated commitments only when
forced by minimum-up-time propagation or system feasibility.

\emph{Windowed no-load variant.}
A gated startup at period~$t$ commits unit~$i$ for at least
$\text{MUT}_i$ periods, some of which may fall beyond the gating
window.
A refined objective attributes no-load cost over the irrevocable
commitment window rather than period-by-period:
\begin{equation}\label{eq:obj_window}
\sum_{i \in \mathcal{I}}
\sum_{t \in \mathcal{T}_i^{\text{gate}}}
\!\left[
  C_i^{SU}\, v_{i,t}
  \;+\;
  \sum_{\tau=t}^{\min\{t+\text{MUT}_i-1,\,T\}}
    \!\! C_i^{NL}\, v'_{i,t}
\right]
+ M \!\sum_t s_t^p,
\end{equation}
where $v'_{i,t} := v_{i,t}\,(1 - v_{i,t}^m)$ isolates reliability
startups from market startups.

\emph{Output.}
Phase~1 produces a candidate commitment schedule
$\hat{\mathbf{u}} = \mathbf{u}^m + \hat{\mathbf{u}}'$, along with
the associated startup and shutdown indicators
$(\hat{\mathbf{v}},\hat{\mathbf{w}})$.

\subsection{Phase~2: Robust Feasibility Verification}
\label{sec:phase2}

Phase~2 takes the commitment decisions from Phase~1 as fixed and
checks whether a feasible dispatch policy exists for all uncertainty
realizations.
Because all binary variables are fixed, Phase~2 is a continuous
optimization problem.

\subsubsection{Uncertainty model}
Let $\mathbf{r} \in \mathbb{R}^K$ represent deviations of renewable
generation from the mean forecast, where $K$ is the number of
uncertain renewable sources.
The uncertainty set is an ellipsoid parameterized by a positive
definite covariance matrix
$\boldsymbol{\Sigma} \in \mathbb{R}^{K \times K}$ and a scalar
budget~$\rho > 0$:
\begin{equation}\label{eq:uset}
    \mathcal{U} \;:=\;
    \bigl\{\, \mathbf{r} \in \mathbb{R}^K \;\big|\;
    \mathbf{r}^\top \boldsymbol{\Sigma}^{-1} \mathbf{r}
    \leq \rho^2 \bigr\}.
\end{equation}
The construction of $\boldsymbol{\Sigma}$ and calibration of~$\rho$
via conformal prediction are detailed in~[paper~2]; here we treat
$(\boldsymbol{\Sigma}, \rho)$ as given inputs.

\subsubsection{Linear decision rules}
Dispatch is modeled as an affine function of the uncertainty
realization:
\begin{equation}\label{eq:ldr}
    p_{i,t}(\mathbf{r})
    \;=\; p_{i,t}^0
    \;+\; \mathbf{z}_{i,t}^\top \mathbf{r},
    \qquad \forall\, i,t,
\end{equation}
where $p_{i,t}^0$ is the nominal dispatch and
$\mathbf{z}_{i,t} \in \mathbb{R}^K$ are the linear decision rule
(LDR) coefficients governing the response of generator~$i$ at
period~$t$ to the uncertainty realization.

\subsubsection{Robust counterpart}
Each operational constraint must hold for every
$\mathbf{r} \in \mathcal{U}$.
For a generic upper-bound constraint
$\mathbf{z}_{i,t}^\top \mathbf{r} \leq b_{i,t}$, the worst-case
over $\mathcal{U}$ yields the second-order cone (SOC) condition
\begin{equation}\label{eq:soc_generic}
    \rho\,
    \bigl\lVert
      \boldsymbol{\Sigma}^{1/2} \mathbf{z}_{i,t}
    \bigr\rVert_2
    \;\leq\; b_{i,t}.
\end{equation}

Applying this to each constraint in $\mathcal{X}^{\text{UC}}$ with
commitment fixed at $\hat{\mathbf{u}}$ gives the Phase~2 formulation:
\begin{subequations}\label{eq:phase2}
\begin{align}
\min_{\mathbf{p}^0,\,\mathbf{Z},\,\mathbf{s}} \quad
& \sum_t s_t
  \label{eq:phase2_obj} \\[3pt]
\text{s.t.}\quad
%--- nominal power balance ---
& \sum_{i} p_{i,t}^0 + s_t = \sum_n d_{n,t},
  \quad \forall\, t,
  \label{eq:phase2_bal_nom} \\
%--- recourse balance ---
& \sum_{i} z_{i,t,k} = 0,
  \quad \forall\, k,t,
  \label{eq:phase2_bal_rec} \\
%--- capacity upper (SOC) ---
& p_{i,t}^0
  + \rho \lVert \boldsymbol{\Sigma}^{1/2} \mathbf{z}_{i,t} \rVert_2
  \leq \bar{P}_i\, \hat{u}_{i,t},
  \quad \forall\, i,t,
  \label{eq:phase2_cap_up} \\
%--- capacity lower (SOC) ---
& \underline{P}_i\, \hat{u}_{i,t}
  \leq p_{i,t}^0
  - \rho \lVert \boldsymbol{\Sigma}^{1/2} \mathbf{z}_{i,t} \rVert_2,
  \quad \forall\, i,t,
  \label{eq:phase2_cap_lo} \\
%--- ramp up (SOC) ---
& (p_{i,t}^0 - p_{i,t-1}^0)
  + \rho \lVert \boldsymbol{\Sigma}^{1/2}
    (\mathbf{z}_{i,t} - \mathbf{z}_{i,t-1}) \rVert_2
  \nonumber\\
& \qquad\qquad
  \leq R_i^U (\hat{u}_{i,t-1} + \hat{v}_{i,t}),
  \quad \forall\, i,t,
  \label{eq:phase2_ramp_up} \\
%--- ramp down (SOC) ---
& (p_{i,t-1}^0 - p_{i,t}^0)
  + \rho \lVert \boldsymbol{\Sigma}^{1/2}
    (\mathbf{z}_{i,t-1} - \mathbf{z}_{i,t}) \rVert_2
  \nonumber\\
& \qquad\qquad
  \leq R_i^D (\hat{u}_{i,t} + \hat{w}_{i,t}),
  \quad \forall\, i,t,
  \label{eq:phase2_ramp_dn} \\
%--- transmission (SOC) ---
& \Big|\sum_n S_n^l
  \Big(\sum_{i \in \mathcal{I}_n} p_{i,t}^0 - d_{n,t}\Big)\Big|
  + \rho \Big\lVert \boldsymbol{\Sigma}^{1/2}
    \sum_{i \in \mathcal{I}_n} S_n^l\, \mathbf{z}_{i,t}
  \Big\rVert_2
  \nonumber\\
& \qquad\qquad
  \leq \bar{F}^l,
  \quad \forall\, l,t,
  \label{eq:phase2_flow} \\
%--- renewable limit (SOC) ---
& p_{j,t}^0
  + \rho \lVert \boldsymbol{\Sigma}^{1/2}
    (\mathbf{z}_{j,t} - \mathbf{e}_{j,t}) \rVert_2
  \leq \mu_{j,t},
  \nonumber\\
& \qquad\qquad
  \forall\, j \in \mathcal{I}^{\text{ren}},\; t,
  \label{eq:phase2_ren} \\
%--- domains ---
& \mathbf{p}^0 \geq 0,\; \mathbf{s} \geq 0.
  \label{eq:phase2_domain}
\end{align}
\end{subequations}

Here $\mathbf{e}_{j,t} \in \mathbb{R}^K$ is the unit vector
selecting the uncertainty dimension corresponding to renewable
source~$j$ at period~$t$, so that the available renewable output
under realization~$\mathbf{r}$ is $\mu_{j,t} + \mathbf{e}_{j,t}^\top
\mathbf{r}$.
Constraint~\eqref{eq:phase2_bal_rec} ensures that the aggregate
dispatch response to any realization sums to zero: when renewable
output deviates, the remaining fleet absorbs the imbalance.

Formulation~\eqref{eq:phase2} is a second-order cone program (SOCP)
with no integer variables.
If the optimal slack satisfies $\sum_t s_t^* = 0$, the Phase~1
commitments are \emph{robust-sufficient}: a feasible dispatch policy
exists for every $\mathbf{r} \in \mathcal{U}$.
If $\sum_t s_t^* > 0$, the commitments are insufficient and the
\emph{robustness gap} $\sum_t s_t^*$ quantifies the worst-case
shortfall in megawatt-hours.


\subsection{Augmentation via Column-and-Constraint Generation}
\label{sec:augment}

When Phase~2 reports a positive robustness gap, the Phase~1
commitments are insufficient under worst-case uncertainty.
We close this gap by iterating between the two phases using
column-and-constraint generation (CCG).

Phase~2 not only evaluates the gap but also identifies a
\emph{worst-case scenario}: the realization
$\mathbf{r}^* \in \mathcal{U}$ at which the dispatch constraints are
most stressed.
This scenario is added to Phase~1 as an additional set of
deterministic constraints, requiring that the commitment schedule be
feasible under both the mean forecast and the worst-case
realization~$\mathbf{r}^*$.
Phase~1 is then re-solved, and the updated commitments are passed
back to Phase~2 for verification.

Formally, at CCG iteration~$\ell$, Phase~1 is augmented with the
scenario set $\mathcal{R}^{(\ell)} = \{\mathbf{r}^{*(1)}, \dots,
\mathbf{r}^{*(\ell)}\}$ collected from prior Phase~2 solves:
\begin{equation}\label{eq:ccg_aug}
  \eqref{eq:phase1} \;\;\text{with additional constraints:}\;\;
  \mathcal{X}^{\text{UC}}(\mathbf{r}^{*(m)}) \;\;\forall\,
  \mathbf{r}^{*(m)} \in \mathcal{R}^{(\ell)},
\end{equation}
where $\mathcal{X}^{\text{UC}}(\mathbf{r})$ denotes the
deterministic UC constraints evaluated at the renewable forecast
$\boldsymbol{\mu} + \mathbf{r}$.
Phase~1 remains a mixed-integer linear program at each iteration
(no SOC constraints); only the number of scenario copies of
$\mathcal{X}^{\text{UC}}$ grows.

\begin{proposition}\label{prop:ccg}
The CCG procedure converges in a finite number of iterations to the
optimal solution of the monolithic robust gated LD-RUC.
\end{proposition}

This follows from the standard finite-convergence result for
two-stage robust optimization with a compact uncertainty set
and finite recourse~[cite Zeng \& Zhao 2013].

In practice, we expect the number of CCG iterations to be small.
When the system has sufficient reserve headroom, Phase~2 may certify
robust feasibility on the first pass, requiring no augmentation at
all.
The robustness gap and iteration count are reported in the case study
as measures of how far the deterministic gated solution lies from
robust sufficiency.

\subsection{Summary}\label{sec:formulation_summary}

The complete LD-RUC procedure is:
\begin{enumerate}
  \item Solve Phase~1~\eqref{eq:phase1}: deterministic gated LD-RUC
    (MIP) with mean forecast.
    Obtain candidate commitments $\hat{\mathbf{u}}$.
  \item Solve Phase~2~\eqref{eq:phase2}: robust feasibility
    verification (SOCP) with fixed $\hat{\mathbf{u}}$.
  \item If $\sum_t s_t^* = 0$: terminate.
    The commitments are robust-sufficient.
  \item If $\sum_t s_t^* > 0$: extract worst-case
    $\mathbf{r}^*$, augment Phase~1
    via~\eqref{eq:ccg_aug}, and return to Step~1.
\end{enumerate}

The separation of commitment (integer, Phase~1) from dispatch
verification (continuous, Phase~2) offers both computational and
structural advantages.
Phase~1 is a standard-sized MIP with the same constraint structure as
a conventional DARUC; the gating set modifies only the objective.
Phase~2 is a SOCP that scales with the number of generators and
periods but involves no combinatorial decisions.


%% ===================================================================
%%  SECTION 5 — GENERALIZATION ACROSS DECISION STAGES
%% ===================================================================
\section{Generalization Across Decision Stages}\label{sec:general}

The gating set~\eqref{eq:gating_set} is parameterized by the
decision stage~$k$ (through $t^{\text{next}}$) and the generator
notification times~$L_i^{SU}$.
By varying these parameters, the LD-RUC framework instantiates
naturally across the operational planning sequence.

\subsection{Multi-Day Reliability Assessment (MDRA)}

The MDRA is an advisory process run several days before the operating
day, with a horizon of four to seven days.
At this stage, $t^{\text{next}}$ corresponds to the first period of
OD$-$1 (the day before the operating day), when the DAMKT will
clear.
Only generators with very long notification times --- those whose
startup lead time exceeds the interval between the MDRA and the
DAMKT --- have non-empty gating sets.
In practice, this restricts the MDRA's binding decisions to large
coal, nuclear, and combined-cycle units requiring multi-day advance
notice.

\subsection{Day-Ahead RUC (DA-RUC)}

The DA-RUC runs after the DAMKT clears for the operating day.
Here $t^{\text{next}}$ is the first period of OD$+$1 (the next
operating day), since the next DAMKT will cover that horizon.
Generators with notification times shorter than 24~hours have gating
sets that cover only the operating day itself; those with longer lead
times may also gate periods on OD$+$1.
This is the primary application of the LD-RUC framework developed in
Section~\ref{sec:phase1}.

\subsection{Intra-Day RUC (ID-RUC)}

Intra-day reliability processes run on a rolling basis during the
operating day.
At each invocation, $t^{\text{next}}$ is the earliest period that can
still be modified by the next intra-day run (typically the current
period plus the process cycle time).
The gating sets shrink as the operating day progresses and more
decisions become irrevocable, leaving only fast-start resources with
non-empty gating windows.

\subsection{Relationship to Standard DARUC}

The standard DARUC formulation~\eqref{eq:daruc_obj} can be viewed as
the limiting case of the gating framework in which
$\mathcal{T}_i^{\text{gate}} = \mathcal{T}$ for all~$i$ --- that is,
every commitment decision is treated as non-deferrable.
This occurs when $t^{\text{next}} \to \infty$ (no subsequent process
exists) or equivalently when $L_i^{SU} \to \infty$ for all
generators.

Conversely, the pure market outcome (no reliability intervention)
corresponds to $\mathcal{T}_i^{\text{gate}} = \varnothing$ for
all~$i$, which occurs when $L_i^{SU} = 0$ (all units can be started
instantaneously) or $t^{\text{next}} = 1$ (the next process can
modify every period).
The gating framework thus spans a continuum between no intervention
and full re-optimization, with the physical notification times
determining the appropriate point on this spectrum.


%% ===================================================================
%%  SECTION 6 — CASE STUDY
%% ===================================================================
\section{Case Study}\label{sec:casestudy}

\subsection{Setup}

The proposed LD-RUC formulation is evaluated on the RTS-GMLC test
case, a three-area system with 73~generating units (thermal, wind,
solar, and hydro) and 120~transmission lines.
Wind forecast data are drawn from the Southwest Power Pool (SPP)
constellation forecast dataset, scaled to the RTS-GMLC wind fleet
capacities.
The ellipsoidal uncertainty set parameters
$(\boldsymbol{\Sigma}, \rho)$ are constructed using the learned
covariance and conformal prediction methodology developed
in~[paper~2].

Notification times $L_i^{SU}$ are assigned to the generator fleet
based on fuel type and capacity:
% TODO: Table of notification time assignments.
% Coal/nuclear: 24-48 hours
% Combined cycle: 8-12 hours
% Combustion turbine: 1-4 hours
% Wind/solar/hydro: 0 hours (no startup lead time)

The DA-RUC decision stage is used as the primary test case, with
$t^{\text{next}}$ set to the first period of OD$+$1 (period~25 for a
48-hour horizon starting at OD hour~1).
All experiments are solved using Gurobi~11 on [hardware description].

\subsection{Standard DARUC vs.\ LD-RUC}\label{sec:results_daruc}

% TODO: Primary comparison results.
% Expected structure:
% - Table: number of reliability commitments (startups) by fuel type
%   for standard DARUC vs LD-RUC
% - Figure: commitment heatmap showing which generators are committed
%   differently between the two approaches
% - Key finding: long-lead units (coal, CC) have identical commitments
%   since they are gated in both formulations; short-lead units (CT)
%   differ significantly
% - Total commitment cost comparison
% - Feasibility verification: confirm no load shedding (s = 0)
%   in both cases

\subsection{Robust Verification Results}\label{sec:results_robust}

% TODO: Phase 2 results.
% - For each test scenario: does Phase 2 pass on the first try?
% - Report pass/fail rate across scenarios
% - When Phase 2 fails: report robustness gap (sum of s_t)
% - Number of CCG iterations required to achieve robust feasibility
% - Key question to answer: how often are deterministic gated
%   commitments already robust-sufficient?
% - Compare final (post-CCG) commitment schedule against Phase 1
%   result: how many additional units were needed?

\subsection{Comparison with Integrated Robust UC}
\label{sec:results_aruc}

% TODO: Compare LD-RUC against the integrated ARUC from paper 2.
% - ARUC jointly optimizes all commitment and dispatch decisions
%   under uncertainty (no gating, no decomposition)
% - Metrics: total cost, commitment count, wind curtailment,
%   computation time
% - Key point: LD-RUC produces fewer out-of-market interventions
%   than ARUC by construction, at the potential cost of higher
%   total system cost
% - The cost gap between LD-RUC and ARUC represents the price of
%   minimal intervention: how much more expensive is the
%   least-distortionary solution?

\subsection{Sensitivity to Notification Times}
\label{sec:results_sensitivity}

% TODO: This section demonstrates the gating set's value directly.
% - Sweep L_i^SU for the fleet (or sweep t^next)
% - For each setting: report number of gated decisions, number of
%   reliability commitments, total cost
% - Figure: commitments and cost vs notification time or t^next
% - Limiting cases as validation:
%   L_i -> 0: no gated decisions, zero reliability cost
%   L_i -> infinity: all decisions gated, recovers standard DARUC
% - Intermediate values: demonstrate smooth transition and
%   quantify the marginal value of each hour of notification time


%% ===================================================================
%%  SECTION 7 — CONCLUSION
%% ===================================================================
\section{Conclusion}\label{sec:conclusion}

This paper introduced a notification-gated formulation for
reliability unit commitment that restricts out-of-market interventions
to the minimal set of commitment decisions that cannot be deferred to
subsequent market or operational processes.
The gating set, derived from physical generator notification times and
the timing of sequential decision stages, provides a principled
criterion for distinguishing non-deferrable reliability actions from
decisions that can be left to the market.

The proposed two-phase solution approach separates the commitment
decision (a deterministic MIP over the gating window) from robust
feasibility verification (a continuous SOCP with fixed commitments).
This decomposition offers computational advantages and provides
diagnostic information: the robustness gap quantifies the worst-case
shortfall of the deterministic solution, and the number of CCG
iterations measures the additional commitment cost attributable to
uncertainty.

The gating framework generalizes across operational horizons.
Multi-day reliability assessment, day-ahead RUC, and intra-day RUC
are all instances of the same formulation with different
parameterizations of the gating set.
Standard DARUC is recovered as a limiting case in which all decisions
are treated as non-deferrable.

% TODO: Summarize numerical findings once case study is complete.
% Expected: significant reduction in reliability commitments,
% Phase 2 passes in most scenarios, small robustness gap when
% it does not, modest CCG iterations.

Future work includes analysis of market efficiency impacts (LMP
changes and uplift costs resulting from reduced reliability
interventions), extension to chance-constrained and distributionally
robust uncertainty models, and application to multi-settlement market
structures where the interaction between reliability processes and
real-time pricing is most consequential.


%% ===================================================================
%%  BACK MATTER
%% ===================================================================

\appendices
\section{Nomenclature}\label{app:nomenclature}

\begin{table}[h]
\renewcommand{\arraystretch}{1.15}
\centering
\begin{tabular}{ll}
\hline
\textbf{Symbol} & \textbf{Description} \\
\hline
\multicolumn{2}{l}{\emph{Sets and indices}} \\
$\mathcal{I}$, $i$ & Generating resources \\
$\mathcal{I}^{\text{ren}}$, $j$ & Renewable generators \\
$\mathcal{I}_n$ & Generators at bus $n$ \\
$\mathcal{N}$, $n$ & Buses \\
$\mathcal{L}$, $l$ & Transmission lines \\
$\mathcal{T}$, $t$ & Operating periods \\
$\mathcal{T}_i^{\text{gate}}$ & Gating set for generator $i$ \\
$\mathcal{U}$ & Uncertainty set \\
$K$, $k$ & Uncertainty dimensions \\
$b$ & Dispatch blocks \\
\hline
\multicolumn{2}{l}{\emph{Parameters}} \\
$C_i^{NL},C_i^{SU},C_i^{SD}$ & No-load, startup, shutdown cost \\
$C_{i,b}^p$ & Block energy cost \\
$\bar{P}_i$, $\underline{P}_i$ & Max/min capacity \\
$\bar{P}_{i,b}$ & Block capacity \\
$R_i^U$, $R_i^D$ & Ramp-up/down rate \\
$\text{MUT}_i$, $\text{MDT}_i$ & Min up/down time \\
$L_i^{SU}$ & Startup notification time \\
$d_{n,t}$ & Nodal demand \\
$\bar{F}^l$ & Line flow limit \\
$S_n^l$ & PTDF entry \\
$\mu_{j,t}$ & Mean renewable forecast \\
$\boldsymbol{\Sigma}$ & Uncertainty covariance matrix \\
$\rho$ & Uncertainty budget \\
$M$ & Slack penalty \\
$u_{i,t}^m$, $v_{i,t}^m$ & DAMKT commitment/startup \\
$t^{\text{next}}$ & First period modifiable by next process \\
\hline
\multicolumn{2}{l}{\emph{Variables}} \\
$u_{i,t}$ & Commitment status (binary) \\
$v_{i,t}$, $w_{i,t}$ & Startup/shutdown (binary) \\
$u'_{i,t}$ & Incremental commitment \\
$p_{i,t}$ & Dispatch (MW) \\
$p_{i,t}^0$ & Nominal dispatch (Phase~2) \\
$\mathbf{z}_{i,t}$ & LDR coefficients (Phase~2) \\
$s_t^p$, $s_t$ & Power balance slack \\
\hline
\end{tabular}
\end{table}

\section*{Acknowledgment}
The authors would like to thank\ldots

%\bibliographystyle{IEEEtran}
%\bibliography{IEEEabrv,../bib/paper}

\end{document}
